
\chapter{\emph{Akka}ros}
\label{sec:akkaros}

\section{\emph{Bakkaros}}
\label{sec:bakkaros}

\section{\emph{Vakkaros}}
\label{sec:vaakkaros}

The main entrance is controlled by a doorman, the
\code{WelcomeMVCActor}. No one enters in the system without his
acknowledgement. The behavior of this mvc actor is to redirect users
to the registration hall or to the main access control zone that leads
into Akkaria, where other actors will navigate and interact with the
user through the system. These actors are the \code{RegisterMVCActor}
and \code{LoginMVCActor}.
%
The session type $w$ is governed by the \code{WelcomeMVCActor}, and is
a typical session type for a delegator, where upon message reception,
possibly after some internal tasks, the delegator transfers user
interaction to other actors, in this case, to the registration or
login mvc actors, with session types $R$ and $S$, respectively.
%
\begin{align}
  W = \branchST{register: R, login: L}
\end{align}
%
\par
%
The finite automata representing this session type is depicted in
Figure~\ref{fig:wmvca-finite-automata}.
%
\begin{figure}[h]
  \centering
  \begin{tikzpicture}[node distance=1.3cm,>=stealth',bend
    angle=45,auto]

    \tikzstyle{every node}=[circle]%
    \tikzstyle{every label}=[red]%

    \tikzstyle{branch}=[circle, draw, minimum size=8mm]%
    \tikzstyle{select}=[diamond, draw]%
    \tikzstyle{pre}=[<-,shorten <=1pt,semithick]%
    \tikzstyle{post}=[->,shorten >=1pt,>=stealth,semithick]%

    \begin{scope}
      % First net
      \node (welcome) at (0,0) [branch] {\small{$w$}};%
      \node (register) at (-1,-2) [branch, gray] {$r$};%
      \node (loginAs) at (1,-2) [branch, gray] {$l$};%

      \path (welcome) edge[post] (register);%
      \path (welcome) edge[post] (loginAs);%
      \path (-1, 0) edge[post] (welcome);%
    \end{scope}

  \begin{pgfonlayer}{background}
    \filldraw [line width=4mm,join=round,black!10] ;
    % (I.north -| R.east) rectangle (w2.south -| e1.west) (I'.north -|
    % l1'.east) rectangle (w2'.south -| e1'.west);
  \end{pgfonlayer}
\end{tikzpicture}
\caption{WelcomeMVCActor Finite automata.}
\label{fig:wmvca-finite-automata}
\end{figure}

%
\par
%
The \code{RegisterMVCActor} governs things from state $r$. He is
responsible to present the user a register form, collect it and try so
persist it internally in the database.
%
Many things can go wrong during the registration process. The input
data provided by the client, \emph{username, password, fullname} is
bound to business rules they must observe. Also, there can be system
failures or any other abnormal situations that may prevent the
registration process to be completed. This means that we need a fault
tolerance mechanism to react adequately to the circumstances.
%
A session type $r$ that includes fault tolerance is given below in
equation \ref{eq:rmvca-register-validate}. Parameters are data that
the user must provide. When receiving such a message, the register
actor enters in a state $v_r$ where the operation is validated.
\begin{align}
  \begin{cases}
    r &= \branchST{register(email, password, fullname): v_r}\\
    v_r &= \selectST{SUCCESSFUL: l, INVALID: r, FAILED: w}
  \end{cases}
          \label{eq:rmvca-register-validate}
\end{align}
%
Validation can result in three possible results: \code{SUCCESSFUL,
  INVALID, FAILED}.
%
If the data provided from client violates any business rule, the
registration will be denied. The actor sends an invalid message to the
caller and resumes to it's initial state $r$.
%
If the data is valid, but a system error or exception occurs during
data persistence in the database, failure will be reported to the
caller and state resumed to $w$.
%
If successful, user interaction will progress to the login state,
under supervision of the \code{LoginMVCActor}.
%
The finite automata representing this interaction is shown in
Figure~\ref{fig:rmvca-finite-automata}.
%
%
\begin{figure}[h]
  \centering
  % Subfigure 1
  \begin{subfigure}
    \centering
    \begin{tikzpicture}[node distance=1.3cm,>=stealth',bend
      angle=45,auto]

      \tikzstyle{every node}=[circle]%
      \tikzstyle{every label}=[red]%

      \tikzstyle{branch}=[circle, draw]%
      \tikzstyle{select}=[diamond, draw]%
      \tikzstyle{pre}=[<-,shorten <=1pt,semithick]%
      \tikzstyle{post}=[->,shorten >=1pt,>=stealth,semithick]%

      \begin{scope}
        % First net

        \node (welcome) at (1,2) [branch, gray] {\small{$w$}};%
        \node (register) at (0,0) [branch] {$r$};%
        \node (login) at (2,0) [branch, gray] {$l$};%

        \path (welcome) edge[post] (register);%
        \path (welcome) edge[post] (login);%

        \node (register) at (0,0) [branch] {\small{$r$}};%
        \node (validate) at (0,-2) [select] {$v_r$};%
        \node (login) at (-1,-4) [branch, gray] {$l$};%
        \node (welcome) at (1,-4) [branch, gray] {$w$};%

        \path (register) edge[post, bend right] (validate);%
        \path (validate) edge[post] (login);%
        \path (validate) edge[post, bend right] (register);%
        \path (validate) edge[post] (welcome);%
        \path (-1, 0) edge[post] (register);%
      \end{scope}

    \end{tikzpicture}
    % \caption{\code{RegisterMVCActor} finite automata}
    \label{fig:rmvca-finite-automata-normal}
  \end{subfigure}
  % Subfigure 2
  \begin{subfigure}
    \centering
    \begin{tikzpicture}[node distance=1.3cm,>=stealth',bend
      angle=45,auto]

      \tikzstyle{every node}=[circle]%
      \tikzstyle{every label}=[red]%

      \tikzstyle{branch}=[circle, draw]%
      \tikzstyle{select}=[diamond, draw]%
      \tikzstyle{pre}=[<-,shorten <=1pt,semithick]%
      \tikzstyle{post}=[->,shorten >=1pt,>=stealth,semithick]%

      \begin{scope}
        % First net

        \node (welcome) at (1,2) [branch, gray] {\small{$w$}};%
        \node (register) at (0,0) [branch] {$r$};%
        \node (login) at (2,0) [branch, gray] {$l$};%

        \path (welcome) edge[post] (register);%
        \path (welcome) edge[post] (login);%

        \node (register) at (0,0) [branch] {\small{$r$}};%
        \node (validate) at (1,-2) [select] {$v_r$};%

        \path (register) edge[post, bend right] (validate);%
        \path (validate) edge[post] (login);%
        \path (validate) edge[post] (register);%
        \path (validate) edge[post] (welcome);%
        \path (-1, 0) edge[post] (register);%
      \end{scope}

    \end{tikzpicture}
    \caption{\code{RegisterMVCActor} normalized finite automata}
    \label{fig:rmvca-finite-automata-normal}
  \end{subfigure}

  \label{fig:rmvca-finite-automata}
  \caption{\code{RegisterMVCActor} session type.}
\end{figure}



%%% Local Variables:
%%% mode: latex
%%% TeX-master: "../main"
%%% End:
 This last automata
introduces two new features: $v_r$, a select state marked by a
diamond, and cycles introduced by the interaction of the verification
state $v_r$ and other system states.
%
In circle states, actors waits for client messages, and on diamond
states the actor communicates it's internal states to clients. Clients
will then synchronize with the actor, providing data until the data is
valid leading the actor to state $l$, or an error occurs bringing the
actor into state $w$. States $l$ and $w$ are governed by other actors,
so we change both state and host actor.
%
State $v_r$ is the state that implements fault tolerance: instead of
terminating the interaction, client and actors behave safely even in
the presence of failures, error or invalid data input.
%
\par
%
The \code{LoginMVCActor} behaves in a similar way
(Figure~\ref{fig:lmvca-finite-automata}). In state $l$, the actor is
expecting a $login$ message with the input data. When the client sends
such message, the actor performs a state transition to $v_l$ where the
result of the operation will be communicated back to the client for
state synchronization. This happen in $v_l$, where fault tolerance
mechanisms are implemented in cases of system failure or violation of
business requirements. If no violation nor failure is detected, and
user redirected to state $u$, the user area under the control of a new
mvc actor: \code{UserMVCActor}, an \emph{avatar} in Akkaria.
%
%
\begin{figure}[h]
  \centering
  % Subfigure 1
  \begin{subfigure}
    \centering
    \begin{tikzpicture}[node distance=1.3cm,>=stealth',bend
      angle=45,auto]

      \tikzstyle{every node}=[circle]%
      \tikzstyle{every label}=[red]%

      \tikzstyle{branch}=[circle, draw]%
      \tikzstyle{select}=[diamond, draw]%
      \tikzstyle{pre}=[<-,shorten <=1pt,semithick]%
      \tikzstyle{post}=[->,shorten >=1pt,>=stealth,semithick]%

      \begin{scope}
        % First net

        \node (welcome) at (1,2) [branch, gray] {\small{$w$}};%
        \node (register) at (0,0) [branch, gray] {$r$};%
        \node (login) at (2,0) [branch] {$l$};%

        \path (welcome) edge[post, gray] (register);%
        \path (welcome) edge[post, gray] (login);%

        \node (register) at (0,0) [branch, gray] {\small{$r$}};%
        \node (validate) at (2,-2) [select] {$v_l$};%
        \node (welcome) at (1,-4) [branch, gray] {$w$};%
        \node (user) at (3,-4) [branch, gray] {$u$};%

        \path (login) edge[post, bend right] (validate);%
        \path (validate) edge[post, bend right] (login);%
        \path (validate) edge[post] (welcome);%
        \path (validate) edge[post] (user);%
        \path (3, 0) edge[post] (login);%
      \end{scope}

    \end{tikzpicture}
    % \caption{\code{RegisterMVCActor} finite automata}
    \label{fig:rmvca-finite-automata-normal}
  \end{subfigure}
  % Subfigure 2
  \begin{subfigure}
    \centering
    \begin{tikzpicture}[node distance=1.3cm,>=stealth',bend
      angle=45,auto]

      \tikzstyle{every node}=[circle]%
      \tikzstyle{every label}=[red]%

      \tikzstyle{branch}=[circle, draw]%
      \tikzstyle{select}=[diamond, draw]%
      \tikzstyle{pre}=[<-,shorten <=1pt,semithick]%
      \tikzstyle{post}=[->,shorten >=1pt,>=stealth,semithick]%

      \begin{scope}
        % First net

        \node (welcome) at (1,2) [branch, gray] {\small{$w$}};%
        \node (register) at (0,0) [branch, gray] {$r$};%
        \node (login) at (2,0) [branch] {$l$};%

        \path (welcome) edge[post, gray] (register);%
        \path (welcome) edge[post, gray] (login);%

        \node (register) at (0,0) [branch, gray] {\small{$r$}};%
        \node (validate) at (1,-2) [select] {$v_l$};%
        \node (user) at (1,-4) [branch, gray] {$u$};%

        \path (login) edge[post] (validate);%
        \path (validate) edge[post, bend right] (login);%
        \path (validate) edge[post] (user);%
        \path (validate) edge[post] (welcome);%
        \path (3, 0) edge[post] (login);%
      \end{scope}

    \end{tikzpicture}
    \caption{\code{RegisterMVCActor} normalized finite automata}
    \label{fig:lmvca-finite-automata-normal}
  \end{subfigure}

  \label{fig:lmvca-finite-automata}
  \caption{\code{RegisterMVCActor} session type.}
\end{figure}



%%% Local Variables:
%%% mode: latex
%%% TeX-master: "../main"
%%% End:

%



\subsection{WelcomeMVCActor}
\label{sec:welcomemvcactor}




\subsection{RegisterMVCActor}
\label{sec:registermvcactor}




\subsubsection{Creating the RegisterMVCActor}
\label{sec:creat-regist}

MVC actor live in the ui and provides a bridge to the backend actors,
forming together a sort of bidirectional communication channel. Users
and views only have direct access to MVC actors, not to business
actors. MVC actors acts as controllers that will internally coordinate
processes and resources, including business actors, to accomplish the
requested task. The source code of \code{RegisterMVCActor} is listed
in Listing~\ref{listing:register-mvc-actor}
%
To create a reference to this actor:
% \lstinputlisting{akkaros/RegisterMVCActor.java}
% \caption{A listing}
\label{listing:register-mvc-actor}
%







The business rules or requirements for the registration process exist.
\begin{itemize}
\item Usernames are email valid addresses; users can register only
  once. The existence of a registration is determined by the existence
  of one and one only \code{Registration} relationship for an account
  whose username matches the input \emph{username}. A username is
  valid if respects:
  \begin{align}
    \forall r = (u, a) \in R,\ u \in U,\ a \in A: a.username \neq
    username
  \end{align}
\item Passwords have to be strong and will be stored encrypted
\item Full names are made of at least two names
\end{itemize}
%

%
After registration there will be new data units in the database. In
particular:
\begin{align}
  \begin{cases}
    \exists^1 u \in U:& u.fullname = fullname\\
    \exists^1 a \in A:& a.username = username\\
    \exists^1 r=(u',a') \in R:& a' = a, u' = u
  \end{cases}
\end{align}
%
\par
%
The behaviour of \code{WelcomeActor} is to coordinate the registration
and login processes. This behaviour is represented by session type
\begin{align}
  \begin{cases}
    W &=\ \branchST{register: R, login: L}\\
    R &=\ \branchST{register(username,password,fullname): V_r}\\
    V_r &=\ \selectST{true: L, false: R, error: W}\\
    L &=\ \branchST{login(username, password): V_l}\\
    V_l &=\ \selectST{true: U, false: L, error: W}
  \end{cases}
          \label{eq:session-type-wa}
\end{align}
%

%
In this example, in $Welcome$ state the actor offers two services,
$register$ and $login$. To access the service clients must send one of
these messages. Exchanging this message will bring the communication
into a new state, either $Register$ or $Login$.
%
In state $Register$ the actor expects a message wrapping user
input. Input validation, business rule validation and error handling
are reported back to the client for state synchronization
($Validate Registration$).
%
If validation and operation were successful, we can go to the login
state $Login$. Registered users can go directly to $Login$.
%
In the login state the actor expects a message wrapping the user
username and password. Input validation, storage, and error are
reported in state $Validate Login$. Upon successful login, the user
can enter the gates of Akkaria in state $U$.
%
\begin{figure}[h]
  \centering
\begin{tikzpicture}[node distance=1.3cm,>=stealth',bend angle=45,auto]

  \tikzstyle{every node}=[circle]
  \tikzstyle{branch}=[circle, draw, minimum size=8mm]
  \tikzstyle{select}=[diamond, draw]
  \tikzstyle{transition}=[->,shorten <=1pt,>=stealth’,semithick]
  \tikzstyle{pre}=[<-,shorten <=1pt,semithick]
  \tikzstyle{post}=[->,shorten >=1pt,>=stealth,semithick]
  \tikzstyle{every label}=[red]

  \begin{scope}
    % First net
\node (i)  at (0,0) [branch] {$W$};
\node (r)  [branch, right of=i] {$R$};
\node (vr) [select,right of=r] {$V_r$};
\node (l)  [branch,right of=vr] {$L$};
\node (vl) [select,right of=l] {$V_l$};
\node (u)  [branch,right of=vl] {$U$};

\path (i) edge[->, blue] (r);
\path (r) edge[->, blue] (vr);
\path (vr) edge[->, blue] (l);
\path (l) edge[->, blue] (vl);
\path (vl) edge[->, blue] (u);

\draw[blue,    post] (i) .. controls (2.2,-1) .. (l);
\draw[red,    post, dashed] (vr) .. controls (1.75,1) .. (i);
\draw[yellow, post, dashed] (vr) .. controls (1.75,0.75) .. (r);
\draw[red,    post, dashed] (vl) .. controls (3,2.2) and (2,2) .. (i);
\draw[yellow, post, dashed] (vl) .. controls (4.25,.75) .. (l);

  \end{scope}

  \begin{scope}[xshift=6cm]
    % Second net
  \end{scope}

  \begin{pgfonlayer}{background}
    \filldraw [line width=4mm,join=round,black!10]
;
%      (I.north  -| R.east)  rectangle (w2.south  -| e1.west)
%      (I'.north -| l1'.east) rectangle (w2'.south -| e1'.west);
  \end{pgfonlayer}
\end{tikzpicture}
  \caption{\code{WelcomeActor} behavior described by finite state
    machine with states $W, R, V_r, L, V_l, U$. Some states are
    represented by circles, and others represented by diamonds. Circle
    states are states were services are offered and ready to be
    consummed by clients; on diamonds states feedback is given back to
    the caller.
}
  \label{fig:sessiontype-wa}
\end{figure}








%%% Local Variables:
%%% mode: latex
%%% TeX-master: "../main"
%%% End:

%
\begin{figure}[h]
  \centering
\begin{tikzpicture}[node distance=1.3cm,>=stealth',bend angle=45,auto]

  \tikzstyle{every node}=[circle]
  \tikzstyle{branch}=[circle, draw, minimum size=8mm]
  \tikzstyle{select}=[diamond, draw]
  \tikzstyle{transition}=[->,shorten <=1pt,>=stealth’,semithick]
  \tikzstyle{pre}=[<-,shorten <=1pt,semithick]
  \tikzstyle{post}=[->,shorten >=1pt,>=stealth,semithick]
  \tikzstyle{every label}=[red]

  \begin{scope}
    % First net
\node (i)  at (0,0) [branch] {$\&$};
\node (r)  [branch, right of=i] {$\&$};
\node (vr) [select,right of=r] {$+$};
\node (l)  [branch,right of=vr] {$\&$};
\node (vl) [select,right of=l] {$+$};
\node (u)  [branch,right of=vl] {$\&$};

\path (i) edge[->, blue] (r);
\path (r) edge[->, blue] (vr);
\path (vr) edge[->, blue] (l);
\path (l) edge[->, blue] (vl);
\path (vl) edge[->, blue] (u);

\draw[blue,    post] (i) .. controls (2.2,-1) .. (l);
\draw[red,    post, dashed] (vr) .. controls (1.75,1) .. (i);
\draw[yellow, post, dashed] (vr) .. controls (1.75,0.75) .. (r);
\draw[red,    post, dashed] (vl) .. controls (3,2.2) and (2,2) .. (i);
\draw[yellow, post, dashed] (vl) .. controls (4.25,.75) .. (l);

  \end{scope}

  \begin{scope}[xshift=6cm]
    % Second net
  \end{scope}

  \begin{pgfonlayer}{background}
    \filldraw [line width=4mm,join=round,black!10]
;
%      (I.north  -| R.east)  rectangle (w2.south  -| e1.west)
%      (I'.north -| l1'.east) rectangle (w2'.south -| e1'.west);
  \end{pgfonlayer}
\end{tikzpicture}
  \caption{\code{WelcomeActor} behavior described by a \emph{session type}.}
  \label{fig:sessiontype-wa-norm}
\end{figure}








%%% Local Variables:
%%% mode: latex
%%% TeX-master: "../main"
%%% End:

%
\begin{figure}[h]
  \centering
\begin{tikzpicture}[node distance=1.3cm,>=stealth',bend angle=45,auto]

  \tikzstyle{every node}=[circle]
  \tikzstyle{branch}=[circle, draw, minimum size=8mm]
  \tikzstyle{select}=[diamond, draw]
  \tikzstyle{transition}=[->,shorten <=1pt,>=stealth’,semithick]
  \tikzstyle{pre}=[<-,shorten <=1pt,semithick]
  \tikzstyle{post}=[->,shorten >=1pt,>=stealth,semithick]
  \tikzstyle{every label}=[red]

  \begin{scope}
    % First net
\node (i)  at (0,0) [branch] {$!$};
\node (r)  [branch, right of=i] {$!$};
\node (vr) [select,right of=r] {$?$};
\node (l)  [branch,right of=vr] {$!$};
\node (vl) [select,right of=l] {$?$};
\node (u)  [branch,right of=vl] {$!$};

\path (i) edge[->, blue] (r);
\path (r) edge[->, blue] (vr);
\path (vr) edge[->, blue] (l);
\path (l) edge[->, blue] (vl);
\path (vl) edge[->, blue] (u);

\draw[blue,    post] (i) .. controls (2.2,-1) .. (l);
\draw[red,    post, dashed] (vr) .. controls (1.75,1) .. (i);
\draw[yellow, post, dashed] (vr) .. controls (1.75,0.75) .. (r);
\draw[red,    post, dashed] (vl) .. controls (3,2.2) and (2,2) .. (i);
\draw[yellow, post, dashed] (vl) .. controls (4.25,.75) .. (l);

  \end{scope}

  \begin{scope}[xshift=6cm]
    % Second net
  \end{scope}

  \begin{pgfonlayer}{background}
    \filldraw [line width=4mm,join=round,black!10]
;
%      (I.north  -| R.east)  rectangle (w2.south  -| e1.west)
%      (I'.north -| l1'.east) rectangle (w2'.south -| e1'.west);
  \end{pgfonlayer}
\end{tikzpicture}
  \caption{\code{WelcomeActor} behavior described by a \emph{session type}.}
  \label{fig:sessiontype-wa-comm}
\end{figure}








%%% Local Variables:
%%% mode: latex
%%% TeX-master: "../main"
%%% End:

%
\begin{figure}[h]
  \centering
\begin{tikzpicture}[node distance=1.3cm,>=stealth',bend angle=45,auto]

  \tikzstyle{every node}=[circle]
  \tikzstyle{branch}=[circle, draw, minimum size=8mm]
  \tikzstyle{select}=[diamond, draw]
  \tikzstyle{transition}=[->,shorten <=1pt,>=stealth’,semithick]
  \tikzstyle{pre}=[<-,shorten <=1pt,semithick]
  \tikzstyle{post}=[->,shorten >=1pt,>=stealth,semithick]
  \tikzstyle{every label}=[red]

  \begin{scope}
    % First net
\node (i)  at (0,0) [branch] {$\&$};
\node (r)  [branch, right of=i] {$\&$};
\node (vr) [select,right of=r] {$+$};
\node (l)  [branch,right of=vr] {$\&$};
\node (vl) [select,right of=l] {$+$};
\node (u)  [branch,right of=vl] {$\&$};

\path (i) edge[->, blue] (r);
\path (r) edge[->, blue] (vr);
\path (vr) edge[->, blue] (l);
\path (l) edge[->, blue] (vl);
\path (vl) edge[->, blue] (u);

\draw[blue,    post] (i) .. controls (2.2,-1) .. (l);
\draw[red,    post, dashed] (vr) .. controls (1.75,1) .. (i);
\draw[yellow, post, dashed] (vr) .. controls (1.75,0.75) .. (r);
\draw[red,    post, dashed] (vl) .. controls (3,2.2) and (2,2) .. (i);
\draw[yellow, post, dashed] (vl) .. controls (4.25,.75) .. (l);

  \end{scope}

  \begin{scope}[yshift=6cm]
    % Second net
\node (i)  at (0,0) [select] {$+$};
\node (r)  [select, right of=i] {$+$};
\node (vr) [branch,right of=r] {$\&$};
\node (l)  [select,right of=vr] {$+$};
\node (vl) [branch,right of=l] {$\&$};
\node (u)  [select,right of=vl] {$+$};

\path (i) edge[->, blue] (r);
\path (r) edge[->, blue] (vr);
\path (vr) edge[->, blue] (l);
\path (l) edge[->, blue] (vl);
\path (vl) edge[->, blue] (u);

\draw[blue,    post] (i) .. controls (2.2,-1) .. (l);
\draw[red,    post, dashed] (vr) .. controls (1.75,1) .. (i);
\draw[yellow, post, dashed] (vr) .. controls (1.75,0.75) .. (r);
\draw[red,    post, dashed] (vl) .. controls (3,2.2) and (2,2) .. (i);
\draw[yellow, post, dashed] (vl) .. controls (4.25,.75) .. (l);
  \end{scope}

  \begin{pgfonlayer}{background}
    \filldraw [line width=4mm,join=round,black!10]
;
%      (I.north  -| R.east)  rectangle (w2.south  -| e1.west)
%      (I'.north -| l1'.east) rectangle (w2'.south -| e1'.west);
  \end{pgfonlayer}
\end{tikzpicture}
  \caption{\code{WelcomeActor} behavior described by a \emph{session type}.}
  \label{fig:sessiontype-wa-norm}
\end{figure}








%%% Local Variables:
%%% mode: latex
%%% TeX-master: "../main"
%%% End:

%
% \input{akkaros/fig-sessiontype-wa-comm}



%%% Local Variables:
%%% mode: latex
%%% TeX-master: "../main"
%%% End:
