
\chapter{Javices}
\label{cha:javices}

A \javice is anything related to Java.
%
\emph{ Um tique ou um vício de programação. Um padrão ou um
  anti-padrão; uma classe ou um comentário bem construídos. Uma boa ou
  uma má prática. Uma obsessão, uma experiência ou um improviso.}
\par
I have a recurring obsession: improvisation. I assume my hypothesis
that all the software needed in the future is still to be done. Since
the future is the future, nothing on it already exists. The future's
software is not yet here nor there. This means we can create it. Then
let's do it! And that is improvisation. So two challenges: develop
future's software units, improvising.
\par
The problem about developing the future's software is that we have to
build it today, and today we only have today's data-structures and
algorithms. To become the pieces of future's software systems they
must be incredibly simple. Only that way they can be reusable and easy
to adapt and compose.
\par
The problem about improvisations is that you have to master the
ability to, out of known elementary pieces, develop unknown,
intriguing, challenging and pleasant compositions. And that is only
possible mastering the individual pieces and the interaction between
them. Improvisers are players and composers at the same time, and this
time is real-time. Live. Here and now, and that is a \emph{timeline}.
\par

\section{Timelines}
\label{sec:timelines}

We are extremely dependent on our conception of time. We organize most
of our activities on the assumption of a linear time flow. Based on a
concept of an experienced \emph{instant} called now, we created a
notion of past, present and future, and tools to include those
concepts in our semantic reasoning system.
%
Timelines are data structure that store the history of a particle or
system of particles, organized by the default time flow, spanning from
$-\infty$ up to $+\infty$.
%
Each point in the timeline is a pair \pair{$e$}{$t_e$} saying that
event $e$ occurred on instant $t_e$. Not so simple. Not so far.
%
\par
Let's start by specifying \emph{particle}, \emph{events} and
\emph{history}.


\subsection{Particles}
\label{sec:particles}

A particle is a minute fragment or quantity of matter, with physical
properties such as volume and mass.
\footnote{https://en.wikipedia.org/wiki/Particle, \date{18 Sep. 2016}}




\subsection{Events}
\label{sec:events}
%
Not all events are instantaneous single events.








%%% Local Variables:
%%% mode: latex
%%% TeX-master: "../main"
%%% End:
