\section{Feature: Vaadin-Akka Integration}
\label{sec:feature-vai}

The architecture on Section~\ref{sec:akkaria} describes a system with
\akka actors integrated in a \vaadin web application.
%
Today's story shows a possible implementation of such integration.
\begin{feature}
  Vaadin-Akka Integration consists in the integration of a \vaadin web
  application with an \akka actor system to provide a highly scalable
  and performant web application. The integration architecture is
  shown in Figure~\ref{fig:vai}.
%

%
  



\begin{figure}[h]
  \centering
  \includegraphics[scale=.45]{figures/VaadinAkkaIntegration.pdf}
  \caption{\vaadin-\akka integration. \vaadin subsystem provides a
    view for clients and communicates with the backend asynchronously
    via the \akka actor system}
  \label{fig:vai}
\end{figure}




%%% Local Variables:
%%% mode: latex
%%% TeX-master: "main"
%%% End:

%

%
  The Vaadin client component contains the views for user
  interaction. The entry point for the Vaadin system is the \code{UI}
  class. The base of our UI hierarchy is the abstract class
  \code{AkkaUI}.
%
  The client communicates with the business components via actors
  implementing the MVC enterprise pattern. They are known as
  MVCActors.
%

%
  \code{MVCActor} coordinates message exchange between view and
  business parts of the application. The view part itself does not
  have access to the business elements. All communication goes as
  asynchronously as possible, exclusively via this MVC actors.
%
  MVC Actors don't have access to the database or any other business
  resource. Their sole purpose is to implement and control the
  communication protocol, coordinating and delegating to business
  actors.
%
  Typically, a mvc actor will process multiple messages on it's
  onReceive method, and then delegate to business actors the work
  units; they implement message ordering and progression of the
  communication protocol.
%

%
  Each \code{BusinessActor} implements a small piece of business
  functionality, possibly using resources like external apis or a
  database. In our case we persist business data in a graph database.
%







%
  \begin{task}
    Create abstract class AkkaUI\\
    % \feature{feature/userUI/akkaUI}{done}
  \end{task}
  \begin{task}
    Rename MyUI to WelcomeUI\\
    % \feature{feature/userUI/akkaUI}{done}
  \end{task}
  \begin{task}
    Make WelcomeUI extend AkkaUI\\
    % \feature{feature/userUI/akkaUI}{done}
  \end{task}
  \begin{task}
    Create class UserUI extends AkkaUI\\
    % \feature{feature/userUI/akkaUI}{done}
  \end{task}
  \begin{task}
    Merge akkaUI with userUI\\
    % \feature{feature/userUI/akkaUI}{todo}
  \end{task}
%

  \begin{task}
    Determine the communication protocol in form of a session type
    specification\\
    % \feature{feature/task/specification}{todo}
  \end{task}
  \begin{task}
    Determine the client-side projection of the communication
    protocol\\
    % \feature{feature/task/specification/client}{todo}
  \end{task}
  \begin{task}
    Determine the server-side projection of the communication
    protocol\\
    % \feature{feature/task/specification/server}{todo}
  \end{task}
%

  \begin{task}
    Create tests that asserts about the behaviour expected by the
    specification, both on client and server sides. This tests should
    verify that:
    \begin{itemize}
    \item All expected messages are received\\
      \feature{feature/task/test/}{todo}
    \item All messages are processed in the order predefined by the
      session type
    \item If termination is mandatory, assert about termination status
    \end{itemize}
  \end{task}



\begin{task}
  Create WelcomeMVCActor, a subclass of MVCActor, as a
  static\footnote{Why \code{static}} inner class of WelcomeUI. This
  actor will implement the MVC pattern of this architecture:
  \begin{task}
    Implement the communication protocol inside the
    \code{onReceive()}, as asynchronously as possible.\footnote{Use
      \code{tell} and \code{forward} actor communication patterns and
      reserve the \code{ask} communication pattern for special cases.}
  \end{task}
  \begin{task}
    Use a \emph{session type based finite state machine} to guide
    communication dealing with message processing order.
  \end{task}
  \begin{task}
    Store incoming messages locally to decide how to proceed and react
    to them when; messages make the fsm to advance in the session type
    performing a state transition
  \end{task}
  \begin{task}
    Define server-side business actors as BusinessActors
  \end{task}
  \begin{task}
    Implement the server-side projection of the asynchronous
    communication protocol in the \code{onReceive()} method.
  \end{task}
\end{task}
%


\end{feature}




%%% Local Variables:
%%% mode: latex
%%% TeX-master: "main"
%%% End:
