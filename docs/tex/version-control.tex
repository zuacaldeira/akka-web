
\section{Version Control}
\label{sec:version-control}


\subsection{Branches}
\label{sec:branches}

Our repository contains the branches described in Table.
%

%
% https://en.wikibooks.org/wiki/LaTeX/Tables#Basic_examples
\begin{table}[h]
  \centering
  \begin{tabular}{l|l|r}
    \hline\hline
    Branch & Description & Grants\\\hline\hline
    \gitbranchOne{master} & Master branch of the akka-web
                            repository & \gituser{admins}\\
    \gitbranch{master}{development} & Development branch & \gituser{developers}\\
    \gitbranch{master}{release} & Release branch & \gituser{ci}\\
    \gitbranch{master}{production} & Production branch & \gituser{users}\\\hline
  \end{tabular}  
  \caption{Github repositories in the \akkaweb project.}
\end{table}
\label{table:branches}


\subsubsection{Branch Naming Convention}
\label{sec:branch-naming-convention}

Let f be a named feature, and t a named task. Every feature and task
is maintained in \github branches of the \akkaweb repository.
%
Branching follows a naming convention:
\begin{itemize}
\item A new feature f will be maintained in branch
  \gitbranch{\gitbranch{development}{feature}}{f}
\item A new task t will be maintained in
  \gitbranch{development}{\gitbranch{feature}{\gitbranch{f}{t}}}
\end{itemize}





%%% Local Variables:
%%% mode: latex
%%% TeX-master: "main"
%%% End:
